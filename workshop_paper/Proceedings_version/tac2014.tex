%
% File acl-hlt2011.tex
%
% Contact: gdzhou@suda.edu.cn
%%
%% Based on the style files for ACL2008 by Joakim Nivre and Noah Smith
%% and that of ACL2010 by Jing-Shin Chang and Philipp Koehn


\documentclass[11pt]{article}
\usepackage{tac2014}
\usepackage{times}		
\usepackage{latexsym}
\usepackage{amsmath}
\usepackage{multirow}
\usepackage{multicol}
\usepackage[hyphens]{url}
\DeclareMathOperator*{\argmax}{arg\,max}
\setlength\titlebox{10.5cm}    % Expanding the titlebox

\title{The Computational Linguistics Summarization Pilot Task}

% Names appear in alphabetical order of last names except Kokil Jaidka
% who we all agree to be the first author of this work
% Participating teams may prefer a different order among the authors 
% from their team. Please let us know in this case.


% Min: please reformat to take up less space.  See for example:
% http://www.comp.nus.edu.sg/~kanmy/papers/airs2014.pdf
% in wing.nus/GITRepositories/zhanghc/AIRS2014.git for LaTeX source formatting
% Min: use \thanks for add in the other authoring information

\author{Kokil Jaidka$^{1}\thanks{\hspace{.2cm}Authors appear in alphabetical order, with the exception of the coordinator of the task, whom is given the first authorship.} $ , Muthu Kumar Chandrasekaran$^{2}$, Beatriz Fisas Elizalde$^{3}$, Rahul Jha$^{4}$, Christopher Jones$^{5}$ \\ {\bf Min-Yen Kan}$^{2,6}${\bf , Ankur Khanna}$^{2}${\bf , Diego Moll\'{a}-Aliod}$^{5}${\bf , Dragomir R. Radev}$^{4}$, \\ {\bf Francesco Ronzano}$^{3}$ and {\bf Horacio Saggion}$^{3}$ \\ 
\\
$^1$ Wee Kim Wee School of Communication \& Information, Nanyang Technological University, Singapore \\
$^2$ Web, IR \/ NLP Group, School of Computing, National University of Singapore, Singapore \\
$^3$ Universitat Pompeu Fabra, Barcelona, Spain\\
$^4$ School of Information, University of Michigan, USA\\
$^5$ Division of Information \& Communication Sciences, Computing Department, Macquarie University, Australia \\
$^6$ Interactive and Digital Media Institute, National University of Singapore, Singapore}

\date{}

\begin{document}
\maketitle
\begin{abstract}
The Computational Linguistics (CL) Summarization Pilot Task was a pilot shared task to use citations to create summaries of scholarly research publications in the domain of computational linguistics.  We describe the background for the task, corpus construction, evaluation methods for the pilot and survey the participating systems and their preliminary results.  The experience gleaned from the pilot will assist in the proper organization of future shared task where difficulties with annotations and scale can be addressed. The annotated development corpus used for this pilot task is available for download 
here: \begin{sloppypar}
\url{https://github.com/WING-NUS/scisumm-corpus}
\end{sloppypar}
\end{abstract}

%%%%%%%%%%%%%%%%%%%%%%%%%%%%%%%%%%%%%%%%%%%%%%%%%%%%%%%%%%%%%%%%%%%%%%
\section{Introduction}

% Min: Ask the other groups if they have any grants that need to be
% acknowledged, which should appear in the acknowledgements here on
% the title page.
This paper describes the evolution and design of the Computational Linguistics (CL) pilot task for the summarization of computational linguistics research papers sampled from the Association of Computational Linguistics' (ACL) anthology. This task was run concurrently with the Text Analysis Conference 2014 (TAC '14), although not formally affiliated with it. This shared task shares the same basic structure and guidelines with the formal TAC 2014 Biomedical Summarization (BiomedSumm) track. A training corpus ``topics'' from CL research papers was released, each comprising a reference paper along with some sampled papers that cited the reference paper. Participants were invited to enter their systems in a task-based evaluation, similar to BiomedSumm.

This paper will describe the participating systems and survey their results from the task-based evaluation.

%%%%%%%%%%%%%%%%%%%%%%%%%%%%%%%%%%%%%%%%%%%%%%%%%%%%%%%%%%%%%%%%%%%%%%
\section{Background}
%Muthu: looks like we need to properly integrate this into an argument 
%/ motivation
%Kokil: Ok. TODO.
Recent work \cite{mohammad2009,abu2011} in scientific document
summarization have used citation sentences (also known as {\it citances}) from citing papers (referred to as {\it CPs}) to create a multi document summary of the reference paper (referred to as the {\it RP}). Their approach followed a three-part process: finding the relevant documents, or the CPs, then selecting sentences which justify the citation in the RP and, finally, generating the summary. In our task, we have created a training corpus comprising human annotations for each of these sub-problems. Human annotators identified the citances in each of (up to) ten randomly sampled CPs for the RP.

Jaidka and Khoo \shortcite{jaidka2013}'s work on summarizing information science articles indicated that most citations clearly
refer to one or more specific discourse facets of the cited paper. Discourse facets indicate the type of information described in the reference span. E.g., ``Aim'' indicates that the citation is about the Aim of the reference paper. From our exploration of the computational linguistics domain, we observed that the discourse facets being cited were usually the aim of the paper, its methods and the results or implications of the work. We applied these observations in annotating discourse facets in our training corpus. 
%moving this into next section
%the Drawing from these observations, we used a different set of discourse facets than BiomedSumm which suit our target domain of %CL papers better. The resultant corpus should be viewed as a development corpus only, such that the community can enlarge it to %a proper shared task with training, development and testing set divisions in the near future.

%%%%%%%%%%%%%%%%%%%%%%%%%%%%%%%%%%%%%%%%%%%%%%%%%%%%%%%%%%%%%%%%%%%%%%
\section{Corpus Construction}
\label{corpus}
A large and important portion of scholarly communication in the domain of computational linguistics is publicly accessible and archived at the ACL Anthology\footnote{\url{http://aclweb.org/anthology/}}. The texts from this archive are also under a Creative Commons license, which allows unfettered access to the published works for any purposes, including downstream research on summarization of its contents.  

We thus view the ACL Anthology as a corpus and randomly sampled published research papers as a base for building our annotated corpus. In selecting materials for resultant corpus from the Anthology, we wanted to enable citation-based summarization. To this end, with consultation from the BiomedSumm organizers, we needed to ensure that the reference paper was cited with appropriate diversity. As of the corpus construction date (18 September 2014), the live Anthology contained approximately 25K publications, exclusive of the third-party papers hosted (i.e., with metadata but without the actual PDF version of the paper) and extraneous files (i.e., front matter and full volumes).  To ensure sufficient opportunity to use citation based summarization, we further removed papers published after and including 2006, leaving 13.8K publications.  We randomized this list to remove any ordering effects.  Starting from the top of the list, we used a combination of Google Web and Scholar searches to approximate the
number of citations (i.e., citing papers (CP)). We retained any paper with over 10 citations.  We vetted the citations to ensure that the citation spread was at least a window of three years, as previous work had indicated that citations over different time periods (with respect to the publication date of the RP) exhibit different tendencies \cite{N13-1067}.

We then used the title search facility of the ACL Anthology Network\footnote{\url{http://clair.eecs.umich.edu/aan/index.php}}
(AAN, February 2013 version), to locate the paper. We inspected and listed all citing papers' Anthology ID, title and year of publication. We note the citation count from Google / Google Scholar and AAN differ substantially.

To report the final list of citing papers, we strived to provide at least three CP for each RP. We defined the following
criteria (in order or priority):
\begin{enumerate}
\item Non-list citation (i.e., at least one citation in the body of
  the CP for the RP not of the form [RP,a,b,c]);
\vspace{-.3cm}
\item The oldest and newest
citations within AAN; and, 
\vspace{-.3cm}
\item Citations from different years. 
\end{enumerate}

We included the oldest and newest citation regardless of criteria 1) and 3), and included a randomized sample of up to 8 additional citing paper IDs that met either criteria 1) and 3). The final list was divided among the annotator group, who are a subset of the authors of this paper, from National University of Singapore and Nanyang Technological University, Singapore. Annotators re-used the resources created for BiomedSumm, which reduced the efforts required; however, a different set of discourse facets were used, which better represented the content of research papers in computational linguistics. The resultant corpus should be viewed as a development corpus only, such that the community can enlarge it to a proper shared task with training, development and testing set divisions in the near future.

%%%%%%%%%%%%%%%%%%%%%%%%%%%%%%%%%%%%%%%%%%%%%%%%%%%%%%%%%%%%%%%%%%%%%%

\section{The CL-Summ Task}
This shared task proposes to solve the same problems posed in the BioMedSumm track, but in the domain of Computational Linguistics. It poses the research problem of building a structured summary of a research paper -- which incorporates facet information (such as Aims, Methods, Results and Implications) from the text of the paper, and ``community summaries'' from its citing papers. \\

\noindent We define the {\it CL-Summ Task} as follows:\\
{\bf Given}: a topic, comprising of the PDF and extracted text of an reference paper (RP) and up to 10 citing papers (CPs).  In each provided CP, the citations to the RP (or citances) have been identified and manually annotated. The information referenced in the RP is also annotated.  \\
{\bf The Challenge}: Output systems are required to perform the following tasks, where the numbering of the task corresponds to those used in the BiomedSumm task.

\begin{itemize}
\item Task 1A: Identify the text span in the RP which corresponds to the citances from the CP. These may be of the granularity of a full   sentence or several sentences (upto 5 sentences), and may be contiguous or not. It may also be a sentence fragment.
\vspace{-.3cm}
\item Task 1B: Identify the discourse facet for every cited text span from a predefined set of facets. Discourse facet is about the type of information described in the reference span. A maximum of 3 reference spans can be marked for 
every citance. In case these spans describe different different discourse facets, the most prevalent discourse facet is annotated.
\end{itemize}
{\bf Evaluation}: Evaluate Task 1A performance by using the ROUGE~\cite{Lin:2004} score to compare the overlap of text spans in
the system output versus the gold standard created by human annotators.


An additional task in BioMedSumm, which was not advertised with this shared task, was:

{\bf Task 2}: Generate a faceted summary of upto 250 words, of the 
reference paper, using itself and the citing papers.
%%%%%%%%%%%%%%%%%%%%%%%%%%%%%%%%%%%%%%%%%%%%%%%%%%%%%%%%%%%%%%%%%%%%%%
\section{Participating teams}
Nine teams expressed an interest in participating in the shared task:
\begin{enumerate}
\item{{\bf clair\_umich$^{*}$} from University of Michigan, Ann Arbor, USA. They proposed a supervised system based on lexical, syntactic and knowledge-based features to calculate similarity scores between sentences in the CPs and the RP.}
\vspace{-.3cm}
\item{{\bf MQ$^{*}$}, from Macquarie University, Australia. They have applied a system developed for the BiomedSumm Task,
  with the exception that they will not incorporate domain knowledge (UMLS). For Task~1A they used similarity metrics to extract the top n sentences from the documents. For Task~1B they used a logistic regression classifier. Next, for the bonus Task~2 they incorporated the distances from Task~1A to rank the sentences.}
\vspace{-.3cm}
\item{{\bf Taln.UPF$^{*}$}, from Universitat Pompeu Fabra, Spain. They have proposed to adapt available summarization tools to scientific texts.}
\vspace{-.3cm}
\item{{\bf CCS2014}, from the IDA Center for Computing Sciences, USA. They proposed to employ a language model based on the sections of the document to find referring text and related sentences in the cited document.}
\vspace{-.3cm}
\item{{\bf IHMC}, A team from IHMC, USA.}
\vspace{-.3cm}
\item{{\bf IITKGP\_sum}, from Indian Institute of Technology, Kharagpur, India. They planned to use citation network structure and
  citation context analysis to summarize the scientific articles.}
\vspace{-.3cm}
\item{{\bf PolyAF}, from The Hong Kong Polytechnic University.}
\vspace{-.3cm}
\item{{\bf TabiBoun14}, from the Bogaziçi University, Turkey. They planned to modify an existing system for CL papers, which uses LIBSVM as a classification tool for facet classification, and plan to use cosine similarity to compare text spans.}
\vspace{-.3cm}
\item{{\bf TXSUMM}, from University of Houston, Texas. Their system consists of applying similarity kernels in an attempt to better
  discriminate between candidate text spans (with sentence granularity). Their system uses an extractive, ranking method.}
\end{enumerate}
The system descriptions and self-reported task results from three teams (denoted with `*' in the above text) are provided in the following sections. 

%%%%%%%%%%%%%%%%%%%%%%%%%%%%%%%%%%%%%%%%%%%%%%%%%%%%%%%%%%%%%%%%%%%%%%
\section{The clair\_umich System --- Comparing Overlap of Word Synsets}
\label{s:umich}
\subsection{Data Preprocessing}

The original source text for the papers in the CL-Summ corpus was not sentence-segmented, which made it difficult to compute evaluation metrics. Accordingly, for each RP, citing sentences were extracted from all its CPs. Each CP sentence was matched to the RP to create the final annotated dataset. 

Given a citing sentence, matching sentences from the RP were compared to the gold standard RP sentences to compute precision / recall. The average number of RP sentences matched for each CP sentence was 1.28 (with standard deviation 1.92). The maximum number of RP sentences matched for a CP sentence was 7. Given that the total number of RP sentences ranged from between 100 to 600, this made it a very challenging classification problem. 

\subsection{Baseline System}

The team first created a baseline system based on TF.IDF cosine similarity. For any citing sentence, the system computed the TF.IDF
cosine similarity with all the sentences in the RP, thus the IDF values differed across each of the 10 RPs.

\subsection{Supervised System}
The supervised system used knowledge-based features derived from WordNet, syntactic dependency based features, and distributional features in addition to the simple lexical features like cosine similarity. These features are described below.

\begin{enumerate}
\item{\bf Lexical Features:} Two lexical features were used -- TF.IDF and the LCS (Longest Common Subsequence) between the citing sentence ($C$) and source sentence $S$, which is computed as:
\vspace{-.3cm}
\begin{eqnarray*}
  \frac{|LCS|}{min(|C|,|S|)}
\end{eqnarray*}
\item{\bf Knowledge Based Features:} Six wordnet-based similarity measures were combined to obtain six sentence similarity features\cite{Banea2012}:path similarity, WUP similarity~\cite{Wu:1994:VSL:981732.981751}, LCH similarity~\cite{leacock1998combining}, Resnik similarity~\cite{Resnik:1995:UIC:1625855.1625914}, Jiang-Conrath similarity~\cite{Jiang97taxonomySimilarity}, and Lin similarity~\cite{Lin:1998:IDS:645527.657297}. 
Using these measures, the similarities between two sentences was computed by creating a set of senses for each of the words in each of the sentences. Given these two sets of senses, the similarity score between citing sentence $C$ and source sentence $S$ was calculated as follows:
\vspace{-.3cm}
\begin{eqnarray*}
  sim_{wn}(C,S) = \frac{(\omega + \sum_{i=1}^{|\phi|}\phi_i) * (2|C||S|)}{|C|+|S|}
\end{eqnarray*}
Here $\omega$ is the number of shared senses between $C$ and $S$. The list $\phi$ contains the similarities of non-shared words in the shorter text, $\phi_i$ is the highest similarity score of the $i$th word among all the words of the lower text \cite{S13-1017}. 
\item{\bf Syntactic Features:} Given a candidate sentence pair, two syntactic dependencies were considered equal if they had the same dependency type, govering lemma, and dependent lemma \cite{S13-1017}. The Stanford parser was used to obtain dependency parses of all the citing sentences and source sentences. Then, if $R_c$ and $R_s$ are the set of all dependency relations in $C$ and $S$, the dependency overlap score was computed using the formula:

\vspace{-.3cm}
\begin{eqnarray*}
  sim_{dep}(C,S) = \frac{2*|R_c \cap R_s| * |R_c||R_s|}{|R_c|+|R_s|}
\end{eqnarray*}
\end{enumerate}
 
%%%%%%%%%%%%%%%%%%%%%%%%%%%%%%%%%%%%%%%%%%%%%%%%%%%%%%%%%%%%%%%%%%%%%%

\section{The MQ System --- Finding the Best Fit to a Citance}
\label{s:mq}
Given the text of a citance, the MQ system ranked the sentences of the reference paper 
according to its similarity to the citance. Every sentence and its citance was modeled 
as a vector and compared using cosine similarity. 

\begin{figure*}
$$
\hbox{MMR} = \arg\max_{D_i\in R\setminus S}\left[\lambda(\hbox{sim}(D_i,Q)) -
(1-\lambda) \max_{D_j\in S} \hbox{sim}(D_i,D_j)\right]
$$  
\begin{quote}
Where:
\begin{itemize}
\item $Q$ is the citance text.
\item $R$ is the set of sentences in the document.
\item $S$ is the set of sentences that haven been chosen in the
  summary so far.  
\end{itemize}
\end{quote}
  \caption{Maximal Marginal Relevance (MMR)}
  \label{fig:mmr}
\end{figure*}

\textbf{Baseline -- Using TF.IDF}
For the baseline system, the TF.IDF of all lowercased words was used, 
without removing stop words (similar to the clair\_umich team). 
Separate TF.IDF statistics were computed for each reference paper, 
using the set of sentences in the paper and the citance text of all 
citing papers.

\textbf{Adding texts of the same topic:}
Since the amount of text used to compute the TF.IDF was relatively little, 
it was presumed that citing papers are of the same topic. Accordingly the 
complete text of all citing papers was added in calculations for the IDF 
component.

\textbf{Adding context:}
In order to  extend the information on each sentence in the reference paper 
%and further add to the approach in Section~\ref{sec:topics}, 
the text from the reference papers was added within a context window of 
20 neighboring sentences to the target sentence. 

\textbf{Re-ranking using MMR:}
The last experiment used Maximal Marginal Relevance (MMR)~\cite{Carbonell:1998} 
to rank the sentences. All sentences were represented as TF.IDF vectors of 
extended information as described in previous paragraph. Then, the final score 
of a sentence was the combination of the similarity with the citance, and 
similarity with the other sentences of the summary, according to the formula 
shown in Figure~\ref{fig:mmr}. A value of $\lambda=0.97$ was chosen.

%%%%%%%%%%%%%%%%%%%%%%%%%%%%%%%%%%%%%%%%%%%%%%%%%%%%%%%%%%%%%%%%%%%%%%
% Muthu2: I think we should move details on the 2 versions of the corpus
% created by UMICH and UPF to the end of 'Corpus Construction' and say 
% "we have 3 versions of the corpus for download and use by the community
% ...."

\section{The Taln.UPF System}

The UPF system performed a heavy sanitization process to overcome encoding issues in the corpus; these issues are detailed in the Section~\ref{sec:limitations}. The following steps were carried out to generate a sanitized version of the corpus that was used as input for any further textual analysis:
\begin{enumerate}
\vspace{-.3cm}
\item {\textbf Automatic PDF-to-text conversion}: Conversion of PDF versions of the paper into text, by means of Poppler\footnote{http://poppler.freedesktop.org/}, a robust PDF-to-text convertor;
\vspace{-.3cm}
\item {\textbf Manual verification of output}: Manual validation of the PDF-to-text conversion errors in order to get a clean textual version of each paper;
\vspace{-.3cm}
\item {\textbf Mapping annotations to clean textual versions}: Inspection of the textual contents of each of the annotation files, and manual mapping of the annotations to the clean textual version of each paper.
\vspace{-.3cm}
\end{enumerate}
This clean, textual version is also shared in the official repository of the CL Corpus with the consent of the participants, as a valuable contribution to the task.
\subsection{Pre-processing / documents’ preparation:}
In the TALN.UPF system, the following text analysis tools were used to pre-process the documents:
\begin{enumerate}
\vspace{-.3cm}
\item \textbf{Sentence splitter and Sentence Sanitizer}: A rule-based sentence splitter and sanitizer was used to identify candidate sentences, and to remove incorrectly annotated sentences
\vspace{-.3cm}
\item \textbf{Tokenizer} and \textbf{POS tagger}: The GATE\footnote{https://gate.ac.uk/ie/annie.html} tool and its ANNIE NLP tools for English were used to tokenize and tag the corpus.
\vspace{-.3cm}
\item \textbf{Sentence TF-IDF vector calculator}: A TF-IDF vector was generated for each sentence. The IDF values of the terms of each document were computed by considering the CPs and RP as a complete corpus.
\vspace{-.3cm}
\end{enumerate}

\subsection{Task 1A: Algorithm for identifying reference paper 
  text spans for each citance}
For Task 1A, the TALN.U	PF system implemented an algorithm to 
map every citation in the CP to one or more (upto three) 
\textbf{reference text spans} from the RP. Sentences from the CP 
that overlapped totally or partially with the citation text were 
selected and referred to as the \textit{citation context} 
(CtxSent1,..., CtxSentN). For every sentence in the RP, the system 
associated a \textit{score} equal to the sum of the TF-IDF vector 
cosine similarities computed between that sentence and each sentence 
belonging to the citation context (CtxSent1,..., CtxSentN). Finally, 
the top N sentences from the RP with the highest \textit{score} were 
chosen as the reference span. Conflicts in choice were resolved by 
preferring sentences that occurred in the same document section in 
the RP. 
% Muthu2: this sentence is not correct. It should be either: 
% "by the prominence of document sections' selected reference spans;"
% or 
% "by the prominence of document sections selected as reference spans;"
Furthermore, scores for referencing sentences were weighted 
by the prominence of document sections selected reference spans; 
% for instance, if there is 6.5\% of all reference spans came from the 
%Abstract section, then the score for a sentences from the Abstract 
for instance, if there 6.5\% of all reference spans were sourced from the
Abstract section, then the score for a sentence from the Abstract 
of the RP is multiplied by 0.065. The system evaluated the performance 
of the algorithm with varying values of N, the number of sentences 
included in the reference span.

\subsection{Task 1B: Algorithm for identifying the discourse facet 
  of the cited text spans}
Task 1B is cast as a sentence classification problem. From the 
corpus, the system selected sentences from the CPs that overlapped 
totally or partially with a manually annotated reference text span 
and they were classified as the discourse facet of the manually 
annotated reference text span. This resulted in a set of 266 cited 
papers' sentences, each characterized by a discourse 
facet~(see Table~\ref{table:task1bSentDistrib}).

\begin{table}[h]\footnotesize
  \begin{center}
  \begin{tabular}{ |l | c |}
    \hline
    Docset & Citing papers \\ \hline
    \textit{Aim} & 46 \\ \hline
    \textit{Hypothesis} & 1 \\ \hline
    \textit{Implication} & 25 \\ \hline
    \textit{Results} & 29 \\ \hline
    \textit{Method} & 165 \\ \hline
    \textbf{TOTAL}: & 266 \\ \hline
    \hline
  \end{tabular}
  \caption{Discourse facet of the sentences of cited papers belonging to a manually annotated reference text span.}
  \label{table:task1bSentDistrib}
  \end{center}
\end{table}

%Muthu2: rewriting this para, please check if this is correct
Every sentence was modelled as a word vector of unigrams, 
bigrams, trigrams and lemmatized versions of all three.
% Muthu2: what is preserving order of stopwords? Doesn't make sense to me.
% they either preserved stopwords or preserved order of the words as they appeared.
% They preserved the order of stopwords as well. 
Order of stopwords was preserved as well.
Sentence classification performance was 
compared across 3 classifiers -- \textit{Naive Bayes (NB)}, \textit{SVM} with linear kernel 
and \textit{Logistic Regression (LR)}. Results from a 10-fold cross validation over the 
set of CP sentences (see Table~\ref{table:task1bSentDistrib}) are shown in Table~\ref{table:task1bAlgorithmComp}. 
Clearly, LR performed the best with an averaged $F_1$ of 0.719.
%Then, three sentence classifiers - \textit{Naive Bayes}, 
%\textit{SVM} with linear kernel and \textit{Logistic Regression} - were built and compared in a 10-fold 
%cross validation over the set of CP sentences (see Table~\ref{table:task1bSentDistrib}). 
%The results of this comparison are shown in Table~\ref{table:task1bAlgorithmComp}. 
%The best sentence classifier obtained an averaged F1 of 0.719. 

\section{Evaluation and Results}
Three teams submitted their self-assessed results, using ROUGE \cite{Lin:2004} 
for task-1A. ROUGE (in specific, the ROUGE-L variant) is a popular evaluation 
method for summarization systems that compares the text output of the system 
against a set of target summaries. Since ROUGE uses the actual contents words,
 and not the offset information of the sentences chosen by the annotation team, 
 we expect non-zero results for cases when a system chooses a sentence that is 
 somewhat similar to (but not identical) to one chosen by annotators.

For Task 1a, the MQ and TALN.UPF systems were unsupervised while clair\_umich 
system was supervised. The former two systems were evaluated over all 10 topics in 
a single run, while clair\_umich reports cross validated performance over the 10 topics. 
For Task 1b, the TALN.UPF system also followed a supervised approach with 10-fold cross 
validation. The ROUGE-L scores have been calculated using the system output of a set 
of selected sentences as the system summary, and comparing their overlap
against the target summaries are the sentences given by the annotators. 

The following paragraphs describe the results for Tasks~1A,~1B, and the
bonus Task~2 which was attempted by the MQ system.

\noindent\textbf{Results for Task 1:} 
Table~\ref{tab:task1av2} shows the ROUGE-L $F_1$ scores of each individual 
reference document from the CL-Summ dataset.

\begin{table*}
\centering
	\begin{tabular}{|r|r|r|r|r|r|r|r|r|}
	\hline
	\multicolumn{3}{|c|}{MQ} & \multicolumn{3}{|c|}{clair\_umich} & \multicolumn{3}{|c|}{TALN.UPF}\\
	\hline
	P & R & $F_1$ & P & R & $F_1$ & P & R & $F_1$\\
	\hline
% 	0.335 & 0.212 & 0.223 & 0.0 & 0.0 & 0.738\\
% Min: no reported R and P for clair\_umich?
% added in by Muthu
% Min2: from Rahul's email On Nov 10, 2014 12:12 AM, ``Rahul Jha'' <rahuljha@umich.edu> wrote:
% Min2: revises F_1 downward from .738 to .487
% Precision: 0.444
% Recall: 0.574
% F-score: 0.487
% 	0.212 & 0.335 & 0.223 & n/a & n/a & 0.738\\
 	0.212& 0.335& 0.223& 0.444& .574& 0.487& 0.194& 0.344& 0.225\\
	\hline
	\end{tabular}
\caption{Task~1A performance for the participating systems expressed as ROUGE-L score
 averaged over all topics.}
\label{tab:task1a}
\end{table*}


\begin{table*}
  \centering
  \begin{tabular}{|l|r|r|r||l|r|r|r|}
  	\hline
	Paper ID & MQ  & clair\_umich & TALN.UPF & Paper ID & MQ System & clair\_umich & TALN.UPF\\
	\hline
	C90-2039 & 0.235 & 0.635 & 0.180 & J00-3003 & 0.196 & 0.559& 0.263\\
	C94-2154 & 0.288 & 0.536 & 0.200 & J98-2005 & 0.101 & 0.344& 0.196\\
	E03-1020 & 0.239 & 0.478 & 0.198 & N01-1011 & 0.221 & 0.498& 0.254\\
	H05-1115 & 0.350 & 0.375 & 0.233 & P98-1081 & 0.200 & 0.367& 0.211\\
	H89-2014 & 0.332 & 0.546 & 0.275 & X96-1048 & 0.248 & 0.535& 0.240\\
	\hline
  \end{tabular}
\caption{Task~1A ROUGE-L F1 scores for individual topics.}
\label{tab:task1av2}
\end{table*}
%change this from the text of Task 2 in my previous commit BUG

\begin{table}[h]\footnotesize
  \begin{center}
  \begin{tabular}{ | l | c | c | c |}
    \hline
    Disc. facet & NB & SVM & \textbf{LR} \\ \hline
    \textit{Aim} & 0.725 & 0.734 & \textbf{0.732} \\ \hline
    \textit{Method} & 0.706 & 0.826 & \textbf{0.828} \\ \hline
    \textit{Implication} & 0.049 & 0.000 & \textbf{0.200} \\ \hline
    \textit{Results} & 0.509 & 0,533 & \textbf{0.533} \\ \hline
    \textit{Hypothesis} & 0.024 & 0.000 &\textbf{ 0.000} \\ \hline
    \textbf{WEIGHED AVG. $F_1$} & 0.623 & 0.698 & \textbf{0.719} \\ \hline
    \hline
  \end{tabular}
  \caption{ Task~1B TALN.UPF: F1-Score comparison of classification algorithms}
  \label{table:task1bAlgorithmComp}
  \end{center}
\end{table}

\textbf{Results for Task 2:} The MQ team performed an additional test to see 
whether information from the citances were useful for building an extractive 
summary, as is the case with the BiomedSumm data \cite{Molla:ALTA2014}. 
They implemented extractive summarization systems with and without information 
from the citances.  The summarizers without information from the citances scored 
each sentence as the sum of the TF.IDF values of the sentence 
elements. They tried the TF.IDF approach described in Section~\ref{s:mq}.

The summarizers with information from the citances scored each candidate sentence 
$i$ on the basis of rank($i$,$c$) obtained in Task 1A, which has values between 0 
(first sentence) and $n$ (last sentence), and represents the rank of sentence $i$ 
in citance $c$:
\vspace{-3mm}
$$
score(i) = \sum_{c\in citances}1-\frac{rank(i,c)}{n}
$$

The summaries were evaluated using ROUGE-L, where the model summaries are the 
abstracts of the corresponding papers. Since paper X96-1048 of the SciSumm 
data did not have an abstract, it was omitted from this experiment.

An example excerpt from a target summary (Abstract) for the reference 
paper J03-3003 is:

\noindent\fbox{\parbox{.47\textwidth}{\it \footnotesize We describe a statistical 
approach for modeling dialogue acts in conversational speech, i.e., speech- act-like 
units such as STATEMENT, QUESTION, BACKCHANNEL, AGREEMENT, DISAGREEMENT, and APOLOGY. 
Our model detects and predicts dialogue acts based on lexical, collocational, and 
prosodic cues, as well as on the discourse coherence of the dialogue act sequence. 
The dialogue model is based on treating the discourse structure of a conversation as 
a hidden Markov model and the individual dialogue acts as observations emanating from 
the model states. Constraints on the likely sequence of dialogue acts are modeled via 
a dialogue act n-gram...  We achieved good dialogue act labeling accuracy (65\% based 
on errorful, automatically recognized words and prosody, and 71\% based on word transcripts, 
compared to a chance baseline accuracy of 35\% and human accuracy of 84\%) and a small 
reduction in word recognition error.}}

The MQ System's output baseline summary for the same reference paper is 20 sentences long; 
below is an excerpt:

\noindent\fbox{\parbox{.47\textwidth}{\it \footnotesize Dialogue Act 
Modeling for Automatic Tagging and Recognition 
of Conversational Speech. In all these cases, DA labels would enrich the available input 
for higher-level processing of the spoken words. The relation between utterances and speaker 
turns is not one-to-one: a single turn can contain multiple utterances, and utterances can 
span more than one turn (e.g., in the case of backchanneling by the other speaker in midutterance). 
The most common of these are the AGREEMENT/ACCEPTS. One frequent example in our corpus was the 
distinction between BACKCHANNELS and AGREEMENTS (see Table 2), which share terms such as ``right'' 
and ``yeah''. Networks compare to decision trees for the type of data studied here. Neural networks 
are worth investigating since they offer potential advantages over decision trees.}
}

Table~\ref{tab:task2v2} shows the breakdown of ROUGE-L $F_1$ scores per
document.

\begin{table*}
  \centering
  \begin{tabular}{|l|r|r|r||l|r|r|r|}
  \hline
	Paper ID & TF.IDF & Task 1A  & Task 1A & 	Paper ID & TF.IDF & Task 1A  & Task 1A  \\
			&	& TF.IDF & MMR & 	&	& TF.IDF & MMR \\
	\hline
	C90-2039\_TRAIN & 0.347 & 0.315 & 0.293 &	J00-3003\_TRAIN & 0.221 & 0.382 & 0.367\\
	C94-2154\_TRAIN & 0.095 & 0.123 & 0.120 & 	J98-2005\_TRAIN & 0.221 & 0.216 & 0.233\\
	E03-1020\_TRAIN & 0.189 & 0.189 & 0.196 &	N01-1011\_TRAIN & 0.187 & 0.268 & 0.284\\
	H05-1115\_TRAIN & 0.134 & 0.306 & 0.321 & 	P98-1081\_TRAIN & 0.241 & 0.210 & 0.206\\
\cline{5-8}
	H89-2014\_TRAIN & 0.294 & 0.319 & 0.320 &	Average & 0.214 & 0.259 & 0.260 \\
	\hline
  \end{tabular}
  \caption{ROUGE-L $F_1$ results for summaries generated by the MQ system.}
  \label{tab:task2v2}
\end{table*}

\section{Discussion}
\vspace{-.3cm}
\subsection{MQ System Performance: BioMedSumm Vs. CL-Summ}
Table~\ref{tab:task1a} compares the results of the MQ system's experiments with 
the SciSumm data, against the results from the BiomedSumm data. In all results 
the systems were designed to return 3 sentences, as specified in the shared task.
 All short sentences (under 50 characters) were ignored, to avoid including 
 headings or mistakes made by the sentence segmentation algorithm.

\begin{table*}
  \centering
  \begin{tabular}{|l|r|r|r|c|r|r|r|c|}
  	\hline
	& \multicolumn{4}{|c|}{CL-Summ} & \multicolumn{4}{|c|}{BiomedSumm}\\
	\hline
	Run & P & R & $F_1$ & CI & P & R & $F_1$ & CI\\
	\hline
    TF.IDF & 0.198 & 0.316 & 0.211 & 0.185--0.240 & 0.326 & 0.273 & 0.279 & 0.265--0.293\\
	topics & 0.201 & 0.324 & 0.217 & 0.191--0.245 & 0.357 & 0.288 & 0.300
	& 0.285--0.316\\
	context & 0.214 & 0.339 & 0.225 & 0.197--0.255 & 0.372 & 0.291 & 0.308
	& 0.293--0.323\\
	MMR & 0.212 & 0.335 & 0.223 & 0.195--0.251 &  0.375 & 0.290 & 0.308 & 0.293--0.323\\ 
	\hline
  \end{tabular}
  \caption{ROUGE-L results of the MQ system runs for Task 1A.}
  \label{tab:task1amq}
\end{table*}

The results show an improvement in both domains, with the exception that MMR does 
not improve over the run that uses TF.IDF over context in CL-Summ, whereas there is 
an improvement in BiomedSumm. The absolute values are better in the BiomedSumm data, 
and looking at the confidence intervals it can be presumed that the difference between 
the best and the worst run is statistically significant in the BiomedSumm data. 
The results in the CL-Summ data are poorer in general and there are no statistically 
significant differences. However, this may be an artifact of the small size of the corpus. 
Overall, the improvement of results in CL-Summ mirrors that of the BiomedSumm data, so it 
can be suggested that on adding more information to the models that compute TF.IDF, the 
results improve. It is expected that alternative approaches, which gather related information 
to be added for computing the vector models will produce even better results. The results 
with MMR appears to be contradictory across the two domains, but the difference is small 
and may not be statistically significant.

\subsection{Tweaking the Parameters - the clair\_umich Baseline}
For any citing sentence, the TF.IDF cosine similarity was computed with all the sentences in the source paper, and any sentences that had a cosine similarity higher than a given threshold were added to the matched sentences. Table~\ref{tab:clairumichbaseline} shows the precision / recall for different values of the cosine threshold.
\begin{table}[h]
  \centering
  \begin{tabular}{|l|r|r|r|r|}
  	\hline
	Similarity & Precision & Recall & $F_1$ \\
	Threshold & & & \\
	\hline
	0.01 & 0.027 & 0.641 & 0.051\\
	0.05 & 0.048 & 0.426 & 0.087\\
	0.1 & 0.060 & 0.235 & 0.095\\
	0.2 & 0.079 & 0.081 & 0.080\\
	0.3 & 0.062 & 0.032 & 0.042\\
	0.4 & 0.022 & 0.085 & 0.012\\
	0.5 & 0.007 &  0.002 & 0.003\\
	\hline
  \end{tabular}
  \caption{Precision/Recall for different values of the cosine threshold 
  			for the baseline clair\_umich system.}
  \label{tab:clairumichbaseline}
\end{table}

The $F_1$ scores seems to reach a maximum at the similarity threshold of 0.1. The recall at the threshold of 0.1 is 0.23, while the precision is only 0.06. This suggests that initial progress can be made on this problem by first removing these spurious matches that have high lexical similarity.
%combined the error analysis systems
\subsection{Error Analysis for the Participating Systems}

Some drawbacks were observed in the approach and evaluation for the MQ system. The example below illustrates the MQ system's output for Task 1a, for the reference paper H89-2014:

\noindent\fbox{\parbox{.47\textwidth}{\it \footnotesize ``The statistical 
methods can be described in terms of Markov models.''                    
``An alternative approach taken by Jelinek, (Jelinek, 1985) is to view the 
training problem in terms of a "hidden" Markov model: that is, only the words 
of the training text are available, their corresponding categories are 
not known.''
``In this regard, word equivalence classes were used (Kupiec, 1989).'' 
The target sentence was: ``The work described here also makes use of a hidden Markov model.'' 
}}
The first sentence of the sample output was very similar to the target sentence. It was not the best match, but it was a close match, and an evaluation metric such as ROUGE would reward it. On the other hand, the second sentence, even though it talked about HMMs, it was not strictly about the approach used by the paper and therefore it should not be rewarded with a good score. However, ROUGE would be too lenient here. This is one of the issues identified by the MQ system in following a purely lexical approach.

In the clair\_umich system, a number of errors made by the baseline system are due to source sentences that match the words but differ slightly in their information content. 

\begin{figure}
\noindent\fbox{\parbox{.47\textwidth}{Citing text: ``use the BNC to build a co-occurrence graph for nouns, based on a  co-occurrence frequency threshold'' \\

\emph{True positives:}
\begin{itemize}
\vspace{-.2cm}
\item{\small ``Following the method in (Widdows and Dorow, 2002), we build a graph in which each node represents a noun and two nodes have an edge between them if they co-occur in lists more than a given number of times.''}
\end{itemize}
\emph{False positives:}
\begin{itemize}
\vspace{-.2cm}
\item{\small ``Based on the intuition that nouns which co-occur in a list are
  often semantically related, we extract contexts of the form Noun,
  Noun,... and/or Noun, e.g. ``genomic DNA from rat, mouse and
  dog''.''}
\vspace{-.2cm}
\item{\small ``To detect the different areas of meaning in our local graphs, we use a cluster algorithm for graphs (Markov clustering, MCL) developed by van Dongen (2000).''}
\vspace{-.2cm}
\item{\small ``The algorithm is based on a graph model representing words and relationships between them.''}
\end{itemize}
}}
\caption{Lexically similar false positive sentences.}
\label{f:2}
\end{figure}
An example is shown in Figure~\ref{f:2}.  Here, even though the false positive sentences contain the same lexical items (nouns, co-occurrence, graph), they differ slightly in the facts presented. Detection of such subtle differences in meaning might be challenging for an automated system.

Another set of difficult sentences is when the citing sentence says something that is implied by the sentence in the RP, as evident in Figure~\ref{f:3}.

\begin{figure}
\noindent\fbox{\parbox{.47\textwidth}{Citing text: ``The line of our argument below follows a proof provided in ... for the maximum likelihood estimator based on nite tree distributions''

\emph{False negatives:}
\begin{itemize}
\item{\small ``We will show that in both cases the estimated probability is tight.''}
\end{itemize}
}}
\caption{Implied example.}
\label{f:3}
\end{figure}
Here, the citing text mentions a proof from the RP, but to match the sentence in the RP, the system needs to understand that the act of showing something in a scientific paper constitutes a proof.

\section{Shortcomings and Limitations}
\label{sec:limitations}
There were several errors and shortcomings of the dataset which were identified in the process of annotating and parsing the corpus for use by the participating systems. 
\begin{itemize}
\item \textbf{Text encoding}: Often, the text was not in UTF-8 format as expected. Some participating teams, like the UPF, solved this by running the universal charset tool provided by Google Code over all the text and annotations in order to determine the right file encoding to use. It was found that some of the files were also in \textit{WINDOWS-1252} and \textit{GB18030} formats, thus making difficult the implementation of an automated homogeneous textual processing pipeline.
\vspace{-.3cm}
\item \textbf{Content}: Some of the older PDF files, when parsed to text or XML, presented several text formatting issues: hyphenation problems, words not separated by blank spaces, page headers and footnotes included in the textual flow, misspelled words, spaces within words, sentences in the wrong place and so on. Unfortunately these errors were OCR parsing errors, and not within our control. We recommended that  participants configure their string matching to be lenient enough to alleviate such problems.
\vspace{-.3cm}
\item \textbf{Errors in citation/reference offsets:} In the original annotations, citation/reference offset numbers were character-based, and relative to an XML encoding which was not shared in the task, and did not match with the offset numbers on the text-only, cleaned version of the document. Although the text versions of the source documents were shared with the intention to help the participants, this often made their tasks more difficult if their system was geared towards numerical and not system matching. A solution was found for reference offsets by revising them to sentence ID numbers based on available XML files from the clair\_umich system's pre-processing stage; however, the citation offsets remain character-based. As a consequence, in order to retrieve the annotated texts, systems like the TALN-UPF manually searched through citing documents to identify the correct offset, and the clair\_umich system created an automatic program to generate sentence offsets. 
\vspace{-.3cm}
\item \textbf{Discontiguous texts}: The use of ``...''  follows the BioMedSumm standard practice of indicating discontiguous texts, meaning that there was a gap between two text spans (citation spans or reference spans). The gap might be because text moves onto a new page. Sometimes there was a formula, page number or figure between two text spans which is not a part of the annotation. However, this notation caused mismatches for sentences which used text from different parts of the same sentence.
\vspace{-.3cm}
\item \textbf{Small corpus:} The corpus comprised only a set of 10 topics, each with upto 10 citing documents. In this small
  dataset, participants were asked to conduct a 10-fold cross validation. The small size of the data set meant that there were no
  statistically significant results, but significance could only be guessed at from the overall trend of the data.
\vspace{-.3cm}
\item \textbf{Errors in file construction:} An automatic, open-source software was used to map the citation annotations from a software, Protege, to a text file.  However, participants identified several errors in the output - especially in cases where there was one-to-many mapping between citations and references.  Besides this, several annotation texts had no annotation ID (Citance Number field).
\end{itemize}

\section{Conclusions}
This paper describes the computational linguistics pilot task for the
faceted summarization of scholarly papers. We describe the three
systems participated in the shared task, and describe the evaluation
of two submitted runs. The teams used versions of TF.IDF as baselines.
The MQ system followed an unsupervised algorithm while
clair\_umich followed a supervised algorithm. For identifying
referenced text spans in reference papers, the best performance was
obtained by clair\_umich's supervised algorithm using lexical,
syntactic and knowledge-based features to calculate the overlap
between sentences in the citation span and the reference
paper. Although no system submitted results for Task 1B, the task
involving identifying the discourse facets of reference text, TALN.UPF
submitted an algorithm which they aim to implement.  Finally, an added
experiment by the MQ system sought to compare baseline summaries of
reference papers, based on a TF*IDF calculation, against gold standard
summaries, comprising the reference paper's abstracts.
 
The clair\_umich system incorporated WordNet synsets for expanding and 
comparing cited text with reference papers, and the use of syntactic features 
further enriched the calculation of overlap. On the other hand, the MQ system 
relied exclusively on reading and comparing texts. Furthermore, their system 
was originally built for the BioMedSumm task -- however, they had to discard 
some domain-specific features for this task. It is possible that the lack of 
domain knowledge, coupled with OCR-related and PDF parsing errors, affected 
the performance of their system in the CL domain.

This task is an initiative for encouraging the development of tools and 
approaches for scientific summarization. It helped us identify 
existing tools and resources to leverage on for this purpose
and also the hindrances which needed to be overcome in order to have a 
systematic and well-coordinated evaluation. However, with results of 
only for two systems, it is not possible to conjecture at what may be the 
better methods for summarizing CL research papers. The 
resources from this task, and its corpus, are freely available 
for interested research groups to experiment with; the corpus is 
first-of-its-kind summarization corpus for compuational linguistics.

The results of the pilot are encouraging: there seems to be ample
interest from the community and it seems possible to answer more
detailed methodological questions with a more detailed analysis and a
larger datasets.  We encourage the community to support a future
proposal to enlarge the pilot to a full scale shared task.  We plan a
systematic annotation of a training, development as well as test sets,
and the availability of more than one gold standard annotation, and
open-sourced tools and resources to support the efforts of
participating teams. We invite the community to join us in this
endeavour with any resources and time they can spare.

%%%%%%%%%%%%%%%%%%%%%%%%%%%%%%%%%%%%%%%%%%%%%%%%%%%%%%%%%%%%%%%%%%%%%%
\bibliographystyle{tac2014}

\section{Acknowledgments}
 This Shared Task is supported in part by the Singapore National
 Research Foundation under its International Research Centre @
 Singapore Funding Initiative and administered by the IDM Programme
 Office.  The authors also acknowledge and thank the BiomedSumm
 organizers -- especially Lucy Vanderwende, Kevin B. Cohen, Prabha
 Yadav, and Hoa Trang Dang -- for lending their expertise in
 organizing this pilot.\\ The {\bf MQ system} was made possible thanks
 to a winter internship granted to Christopher Jones by the Department
 of Computing, Macquarie University. \\ The {\bf clair\_umich system}
 wishes to acknowledge the helpful suggestions of Ben King, Mohamed
 Abouelenien and Reed Coke. \\ The {\bf TALN.UPF} system is supported
 by the project Dr. Inventor (FP7-ICT-2013.8.1 611383), programa
 Ram\'on y Cajal 2009 (RYC-2009-04291), and the project
 TIN2012-38584-C06-03 Ministerio de Econom\'{\i}a y Competitividad,
 Secretar\'{\i}a de Estado de Investigaci\'on, Desarrollo e
 Innovaci\'on, Spain.

\bibliography{tac2014}

\end{document}
