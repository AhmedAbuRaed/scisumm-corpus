%
% File acl-hlt2011.tex
%
% Contact: gdzhou@suda.edu.cn
%%
%% Based on the style files for ACL2008 by Joakim Nivre and Noah Smith
%% and that of ACL2010 by Jing-Shin Chang and Philipp Koehn


\documentclass[11pt]{article}
\usepackage{tac2014}
\usepackage{times}		
\usepackage{latexsym}
\usepackage{amsmath}
\usepackage{multirow}
\usepackage{url}
\DeclareMathOperator*{\argmax}{arg\,max}
\setlength\titlebox{10.5cm}    % Expanding the titlebox

\title{The Computational Linguistics Summarization Pilot Task}

% Names appear in alphabetical order of last names except Kokil Jaidka
% who we all agree to be the first author of this work
% Partiicipating teams may prefer a different order among the authors 
% from their team. Please let us know in this case.


% Min: please reformat to take up less space.  See for example:
% http://www.comp.nus.edu.sg/~kanmy/papers/airs2014.pdf
% in wing.nus/GITRepositories/zhanghc/AIRS2014.git for LaTeX source formatting
% Min: use \thanks for add in the other authoring information

\author{Kokil Jaidka$^\thanks{\hspace{.2cm}Authors appear in alphabetical order, with the exception of the coordinator of the task, whom is given the first authorship.}1$, Muthu Kumar Chandrasekaran$^{2}$, Rahul Jha$^{3}$, Christopher Jones$^{4}$ \\ {\bf Min-Yen Kan}$^{2,5}$, Ankur Khanna$^{2}$, 
  Diego Molla-Aliod$^{4}$, Dragomir R. Radev$^{3}$, Francesco Ronzano$^{6}$ and Horacio Saggion$^{6}$}

%%   %Francesco Ronzano $\pm$ \\
%%   %$\pm$ \small Universitat Pompeu Fabra \\
%%   %\small Barcelona \\ \AND
%%   %Horacio Saggion $\pm$ \\

%%\author {Kokil Jaidka\thanks{\hspace{.2cm}Authors appear in alphabetical order, with the exception of the coordinator of the task, whom is given the first authorship.}$^{*}$ \\
%%   \small *Wee Kim Wee School of  \\
%%   \small Communication \& Information \\
%%   \small Nanyang Technological University \\
%%   \small {\tt koki0001@e.ntu.edu.sg} \And  
%%   Muthu Kumar Chandrasekaran**$\dagger$ \\
%%   \small **Dept. of Computer Science \\
%%   \small School of Computing \\
%%   \small National University of Singapore \\
%%   \small {\tt muthu.chandra}\\
%%   \small {\tt @comp.nus.edu.sg}  \And  
%%    Rahul Jha $\ddagger$ \\
%%   $\ddagger$\small Univeristy of Michigan \\
%%   \small Ann Arbor, MI 48109  \\
%%   \small {\tt rahuljha@umich.edu} \AND  
%%   %ROW 2  
%%   Christopher Jones+ \\
%%   \small +Division of Information \& \\
%%   \small Communication Sciences \\
%%   \small Computing Department\\
%%   \small Macquarie University \\ 
%%   \small{\tt christopher.jones@mq.edu.au} \And
%%   Min-Yen Kan**$\dagger$ \\
%%   \small {\tt kanmy@comp.nus.edu.sg} \And  
%%   Ankur Khanna$\dagger$ \\
%%   \small $\dagger$ Web, IR \/ NLP Group \\
%%   \small School of Computing \\
%%   \small National University of Singapore \\
%%   \small {\tt khanna89ankur@gmail.com} \AND  
%%   Diego Molla-Aliod+ \\
%%   \small{\tt diego.molla-aliod@mq.edu.au} \And
%%   Dragomir Radev $\ddagger$ \\
%%   \small {\tt radev@umich.edu} \And  
%%   %Francesco Ronzano $\pm$ \\
%%   %$\pm$ \small Universitat Pompeu Fabra \\
%%   %\small Barcelona \\ \AND
%%   %Horacio Saggion $\pm$ \\
%%   %\small {\tt horacio.saggion@upf.edu}
%%  }
 
%rahul jha clair_umich
%dragomir radev clair_umich
%kokil jaidka ntu
%muthu nus
%ankur nus
%Francesco Ronzano, TALN Group, Universitat Pompeu Fabra, Barcelona - EMAIL: francesco.ronzano@upf.edu
%Horacio Saggion, TALN Group, Universitat Pompeu Fabra, Barcelona - EMAIL: horacio.saggion@upf.edu
%christopher jones and diego, mcq univ


\date{}

\begin{document}
\maketitle
\begin{abstract}
The Computational Linguistics (CL) Summarization Pilot Task was a
pilot shared task to use citations to create summaries of scholarly
research publications in the domain of computational linguistics.  We
describe the background for the task, corpus construction, evaluation
methods for the pilot and survey the participating systems and their
preliminary results.  The experience gleaned from the pilot will
assist in the proper organization of future shared task where
difficulties with annotations and scale can be addressed.
\end{abstract}

\section{Introduction}

% Min: Ask the other groups if they have any grants that need to be
% acknowledged, which should appear in the acknowledgements here on
% the title page.
This paper describes the evolution and design of the CL pilot
task\footnote{This research is support in part by the Singapore
  National Research Foundation under its International Research Centre
  @ Singapore Funding Initiative and administered by the IDM Programme
  Office.} for the summarisation of computational linguistics
research papers sampled from the Association of Computational
Linguistics' (ACL) anthology.  This task was concurrently publicized
with TAC 2014, although it is not formally affiliated with the same,
and shares its basic structure and guidelines with the more formal TAC
2014 Biomedical Summarization (BiomedSumm) track. A development corpus
of training ``topics'' from CL research papers was released, each
comprising a main, cited paper along with associated citing
papers. Participants were invited to enter their systems in a
task-based evaluation, similar to BiomedSumm.

This paper will describe the participating systems and survey their 
results from the task-based evaluation.

\section{Background}
Recent works \cite{mohammad2009};\cite{abu2011} in scientific document 
summarisation have used citation sentences or citances from citing  papers 
to create a multi document summary of the reference paper (RP). 
%Muthu: looks like we need ro properly integrate this into an argument / motivation

As proposed by \cite{vu2010}; \cite{hoang2010} the summarisation can be 
decomposed into finding the relevant documents, in this case, the citing 
papers (CPs), then selecting sentences from those papers that cite and 
justify the citation and finally generate the summary. To help tackle each 
of these subproblems, we created gold standard datasets where human annotators 
identify the citances in each of about 10 randomly sampled citing papers for 
the RP.

%Muthu: moving annotation information down to data

A pilot study conducted in the information science domain indicated that 
most citations clearly refer to one or more specific aspects of the cited 
paper \cite{jaidka2013}. For computational linguistics, we identified that 
the discourse facets being cited were usually the aim of the paper, methods 
followed and the results or implications of the work. Accordingly, we used a 
different set of discourse facets than BiomedSumm which suit CL papers better. 
Please note that this is a development corpus and only a training set is 
available for use now. Although, we plan to release a test set of documents 
for next year's evaluation, we report k fold cross-validated performance over 
the 10 documents for the systems registered for participation.

\section{Corpus Construction}
\label{corpus}
The CL community uses the ACL Anthology Reference Corpus \cite{bird2008} 
to evaluate and report performance of systems. To support further research 
in scientific document summarisation among the CL community and beyond we 
% Muthu: just to confirm, we used the live Anthology and not ACL ARC right?
plan to build a manually annotated corpus using research papers sampled 
from the ACL Anthology. As first steps towards this goal we created an 
annotated development corpus by randomly sampling 10 documents from the 
ACL anthology. 

We now describe our construction in detail. From the current, live ACL
 Anthology, there are approx 26K (exactly 25961) individual papers that 
 are in the Anthology (as of 18 September 2014; including 
ones staged for publication but not actually published yet). These only include 
files that we have PDFs hosted (e.g., LREC is not represented as we don't hold 
these PDFs in the Anthology, just their metadata). This number is approximate 
as there are some files that are not publications (frontmatter, author 
indices) that are included. Culling all files before and including 2006, we get 
13.8K (13838) publications, which include conference and journal articles. We 
randomized this list to remove any ordering affects. Starting from the top of 
the list, we used Google search 
\footnote{http://www.google.com.sg}
on (18 September 2013) to search for the 
publication - first by using its Anthology ID as a query 
(e.g.,``H89-2014.pdf'') and not productive, re-queried by the title of the 
paper as a string (e.g., ``Some Experiments with a Naive Bayes WSD System'').
We look for a Scholar search result that shows \# of citations. This was an 
approximation. We kept any paper with over 10 citations. Some papers had 
some similar versions that presented different citation rates; however, 
all of these were dropped anyways due to low citation rate.
We vetted the citations from Google Scholar
\footnote{http://www.scholar.google.com}
 for the citation spread being over 3 years as per citing papers' year 
 of publication (as in Google Scholar). 
We did not attempt to check for publication years that Google Scholar doesn't 
report for some publications. We check only the earliest range manually to 
ensure that the citation is the correct one, as there are usually many examples 
of later citations. We also vetted that there were at least 10 of these citing 
sources available for download. Some candidates were dropped due to the few 
amount of available files that were freely downloadable from the Web.
% Muthu: Ok I am a bit lost on the purpose of the following step
% if not substantial can we skip this
We ran a title search to find the paper in ACL Anthology Network  
\footnote{http://clair.eecs.umich.edu/aan/index.php}
(AAN, February 2013 version). We inspected and listed the citing papers 
(incoming citations) Anthology ID, title and year where the citing papers 
were given in reverse chronological order. Note the citation count from 
Google / Google Scholar and AAN (Feb 2013 release) will differ substantially.

To report the final list of citing papers, we strived to provide at least 
3 citing papers for each paper. To do so, we defined the following criteria 
in order or priority):
\begin{enumerate}
\item Non-list citation (i.e., at least one citation for the
target paper not of the form [X,a,b,c]); 
\item The oldest and newest
citations within AAN; and, 
\item Citations from different years. 
\end{enumerate}

We thus provided the oldest and newest citation regardless of criteria 
1) and 3) and included a randomized sample of up to 8 additional citing 
paper IDs that met either criteria 1) and 3). To do this, we started by 
first randomizing the list of citing papers and enumerating up to 8 
additional citing papers. At this point, the citing papers were listed 
as either ``old(est)'', ``new(est)'', 1-8 (additional citing papers),  
or ``sub'' (substitute backup citing paper in case of disqualification of 
one of the 1-8 additional papers due to criterion 1). We combed through the 
lists of the additional 1-8 citing papers per target paper, from the top to 
the bottom of the randomized list.  We unilaterally collected the top most 
oldest and top most newest paper, in case of ties.  For the remaining (up to) 
8 papers, we examined the citing paper's PDF file to ensure that at least one 
citation made was of a single citation format (e.g., [X] and not [X,a,b,c]). 
Any invalidated files were marked with "list" (citation) mark. The resulting 
final list was divided among the annotators to add human annotations using the 
same scheme used by annotators of the BiomedSumm track's corpus.

Given a reference paper and up to 10 citing papers, annotators from 
National University of Singapore and Nanyang Technological University 
were instructed to find citations to the reference paper (RP) in the 
Citing Papers (CP). Annotators followed instructions used for annotation of corpus 
for the BiomedSumm to encourage cross participation across the two tasks. 
Specifically, the citation text, citation marker, reference text, and 
discourse facet were marked for each citation of the RP found in the CP.

\section{The Task}
In this task, we explore a new form of structured summary: a faceted summary 
of the traditional self-summary (the abstract) and the community summary 
(the collection of citances).  As a third component, we propose to group the 
citances by the facets of the text that they refer to. We propose that by 
identifying first, the cited text span, and second, the facet of the paper 
(Aim, Method, Result or Implication), we can create a faceted summary of the 
paper by clustering all cited/citing sentences together by facet.

The CL-Summ Task is defined as follows:

Given: A topic consisting of a Reference Paper (RP) and upto ten Citing Papers 
(CPs) that all contain citations to the RP. In each CP, the text spans 
(i.e., citances) have been identified that pertain to a particular citation to 
the RP.

Task 1a: For each citance, identify the spans of text (cited text spans) in the 
RP that most accurately reflect the citance. These are of the granularity of a 
sentence fragment, a full sentence, or several consecutive sentences 
(no more than 5).

Task 1b: For each cited text span, identify what facet of the paper it belongs to, 
from a predefined set of facets.

Evaluation: Task 1 will be scored by overlap of text spans in the system output vs 
the gold standard created by human annotators.

\section{Participating teams}
The following 9 teams expressed an interest in participating while BUG 3 teams have 
submitted their findings which are included in this paper:

\begin{enumerate}
\item{Taln.UPF, from Universitat Pompeu Fabra, Spain. They have proposed to adapt available summarisation tools to scientific texts.}
\item{clair\_umich from University of Michigan, Ann Arbor, USA.}
\item{IITKGP\_sum, from Indian Institute of Technology, Kharagpur, India. They plan to use citation network structure and citation context analysis to summarise the scientific articles.}
\item{CCS2014, from the IDA Center for Computing Sciences, USA. They will employ a language model based on the sections of the document to find referring text and related sentences in the cited document.}
\item{TabiBoun14, from the Boğaziçi University, Turkey. They plan to modify an existing system for CL papers, wherein they use LIBSVM as a classification tool for face classification. They also plan to use the cosine similarity metric to compare text spans.}
\item{PolyAF, from The Hong Kong Polytechnic University.}
\item{TXSUMM, from University of Houston, Texas. Their system consists of applying similarity kernels in an attempt to better discriminate between candidate text spans (with sentence granularity). They are using an extractive procedure with ranking algorithms.}
\item{MQ, from Macquarie University, Australia. They plan to use the same system that was used for the BiomedSumm track, with the exception that they will not incorporate domain knowledge (UMLS). For task 1a they will use similarity metrics to extract the top n sentences from the documents. For task 1b we will use a logistic regression classifier. For task 2 we will incorporate the distances from task 1 to rank the sentences.}
\item{A team from IHMC, USA}

\end{enumerate}

\section{The MQ System - Finding the Best Fit to a Citance}

Given the text of a citance, the MQ system ranks the sentences of the reference paper according to its similarity to the citance. Every sentence and its citance was modeled as a vector and compared using cosine similarity. The team experimented with different forms of representing the information in the vectors, and different forms of using the similarity scores to perform the final sentence ranking.


\begin{figure*}
$$
\hbox{MMR} = \arg\max_{D_i\in R\setminus S}\left[\lambda(\hbox{sim}(D_i,Q)) -
(1-\lambda) \max_{D_j\in S} \hbox{sim}(D_i,D_j)\right]
$$  
\begin{quote}
Where:
\begin{itemize}
\item $Q$ is the citance text.
\item $R$ is the set of sentences in the document.
\item $S$ is the set of sentences that haven been chosen in the
  summary so far.  
\end{itemize}
\end{quote}
  \caption{Maximal Marginal Relevance (MMR)}
  \label{fig:mmr}
\end{figure*}


\subsection{Baseline - Using \emph{tf.idf}}
\label{sec:tfidf}
For the baseline system, the \emph{tf.idf} of all lowercased words was used, 
without removing stop words. Separate \emph{tf.idf} statistics were computed 
for each reference paper, using the set of sentences in the paper and the citance 
text of all citing papers.

\subsection{Adding texts of the same topic}
\label{sec:topics}
Since the amount of text used to compute the \emph{tf.idf} in 
Section~\ref{sec:tfidf} was relatively little, the complete text of all citing 
papers was added, under the presumption that citing papers are presumably of 
the same topic as the reference paper. By adding this text we hope to include 
complementary information that can be useful for extending and computing 
the \emph{idf} component. 

\subsection{Adding context}
\label{sec:context}
In order to  extend the information of each sentence in the reference paper 
and further add to the approach in Section~\ref{sec:topics}, the text from 
the reference papers was added within a context window of 20 sentences by 
including the neighouring sentences, centered in the target sentence.

\subsection{Re-ranking using MMR}
\label{sec:mmr}
The last experiment used Maximal Marginal Relevance (MMR) \cite{Carbonell:1998} 
to rank the sentences. All sentences were represented as \emph{tf.idf} vectors 
of extended information as described in Section~\ref{sec:context}. Then, the 
final score of a sentence was the combination of the similarity with the 
citance and similarity of the other sentences of the summary according to the 
formula shown in Figure~\ref{fig:mmr}. A value of $\lambda=0.97$ was chosen.


\section{The clair\_umich System - Comparing Overlap of Word Synsets}
\subsection{Data Preprocessing}

The original SciSumm corpus contained data for 10 papers sampled from the ACL 
Anthology. For each of these papers, citing sentences were extracted from all 
its citing papers. Each citing sentence was then matched to a text segment in 
the original paper creating the final annotated dataset. The original source 
text for the papers in the SciSumm corpus was not sentence segmented, which 
made it difficult to compute evaluation metrics.

Data preprocessing of the SciSumm corpus was done in the following way - First, 
sentences from the reference papers were segmented and then matched to each of 
these source sentences to the SciSumm annotation files. This yielded a fixed 
set of source sentences from the original files, a subset of which were 
matched to each citing sentence. In this way, given a citing sentence, 
matching sentences from the source paper were compared to the gold standard 
sentences matched from the source paper and compute precision / recall. 

The average number of source sentences matched for each citing sentence was 
1.28 (with standard deviation 1.92). The maximum number of source sentences 
matched for a citing sentence was 7. Given that the total number of source 
sentences for papers ranged from between 100 to 600, this made it a very 
challenging classification problem. 


\subsection{Baseline System}

Like the MQ system, the team first created a baseline system based on TF*IDF 
cosine similarity. For any citing sentence, the system computed the TF*IDF 
cosine similarity with all the sentences in the source paper, thus the IDF 
values differed across each of the 10 source papers. 

\subsection{Supervised System}
% Muthu: Does WorNet need to be cited? or atleast the princeton web service
% could be added if that is the same version beoing used for this work
The supervised system used knowledge based features derived from WordNet,
syntactic dependency based features, and distributional features in addition 
to the simple lexical features like cosine similarity. These features are 
described below.

\paragraph{Lexical Features} Two lexical features were used - tf*idf and the 
LCS (Longest Common Subsequence) between the citing sentence ($C$) and source 
sentence $S$, which is computed as:

\begin{eqnarray*}
  \frac{|LCS|}{min(|C|,|S|)}
\end{eqnarray*}

\paragraph{Knowledge Based Features} The system also used set of features 
based on Wordnet similarity. Six wordnet based word similarity measures were 
combined to obtain six knowledge based sentence similarity features using the 
method proposed in \cite{Banea2012}. The wordnet based word similarity 
measures used are path similarity, WUP 
similarity~\cite{Wu:1994:VSL:981732.981751}, 
LCH similarity~\cite{leacock1998combining}, 
Resnik similarity~\cite{Resnik:1995:UIC:1625855.1625914}, Jiang-Conrath 
similarity~\cite{Jiang97taxonomySimilarity}, and Lin 
similarity~\cite{Lin:1998:IDS:645527.657297}. 

Given each of these similarity measures, the similarities between two 
sentences was computed by first creating a set of senses for each of the words 
in each of the sentences. Given these two sets of senses, the similarity 
score between citing sentence $C$ and source sentence $S$ was calculated 
as follows:

\begin{eqnarray*}
  sim_{wn}(C,S) = \frac{(\omega + \sum_{i=1}^{|\phi|}\phi_i) * (2|C||S|)}{|C|+|S|}
\end{eqnarray*}

Here $\omega$ is the number of shared senses between $C$ and $S$. The list 
$\phi$ contains the similarities of non-shared words in the shorter text, 
$\phi_i$ is the highest similarity score of the $i$th word among all the 
words of the lower text \cite{S13-1017}. 

\paragraph{Syntactic Features} An additional feature based on similarity of 
dependency structures was used, by applying the method described in 
\cite{S13-1017}. The Stanford parser was used to obtain dependency parse all 
the citing sentences and source sentences. Given a candidate sentence pair, 
two syntactic dependencies were considered equal if they have the same 
dependency type, govering lemma, and dependent lemma. If $R_c$ and $R_s$ are 
the set of all dependency relations in $C$ and $S$, the dependency overlap 
score was computed using the formula:

\begin{eqnarray*}
  sim_{dep}(C,S) = \frac{2*|R_c \cap R_s| * |R_c||R_s|}{|R_c|+|R_s|}
\end{eqnarray*}


\section{The UPF System}
\subsection{Pre-processing / documents’ preparation:}
The UPF system carried out the following set of preprocessing steps on the 
papers of each topic:
\begin{itemize}
\item{Sentence segmentation:}To identify candidate sentences that will be 
validated or rejected in the following pre-processing steps;
\item{Tokenizer and POS tagger:}Using the open-source GATE software
\item{Sentence sanitizer:} To remove incorrectly annotated sentences, 
relying on a set of rules and heuristics;
\item{Document structural analyzer:} To classify each sentence as belonging 
to one of the following document structural categories: Abstract, 
Introduction, Result\_Discussion, Experimental\_Procedure, 
Supplemental\_Data, Material\_Method, Conclusion, Acknowledgement\_Funding, 
and Reference;
\item{Sentence TFIDF vector computation:} To associate to each sentence a 
TFIDF vector where the IDF values are computed over all the papers of the 
related topic (up to 10 citing paper and one reference paper).
\end{itemize}

\subsection{ Algorithm for identifying reference paper text spans 
				for each citance}
\begin{itemize}
\item{For each citance its global citatance context span was considered as the union 
of the citance context spans} marked by human annotators (in this case, there 
was only one available human annotation, so no union was required).
\item{Then we select the sentences of the citing paper} that overlap totally 
or partially the global citatance context span; these sentences are referred 
to as the citance context sentences (CtxSent1,..., CtxSentN),
\item{We characterize the citance by the document structural category 
associated with most of its citance context sentences 
(CtxSent1,..., CtxSentN)}. In case of tie in the number of occurrences of 
document structural categories among all the citance context sentences, we 
choose the document structural category that is most frequent in the citing 
paper. In case of persisting tie, we select the document structural category 
that is most frequent in the whole set of citing and reference papers.
\item{We associate to each reference paper sentence (RefSent) a score} equal 
to the sum of its TFIDF vector cosine similarity with each citance context 
sentence (CtxSent1,..., CtxSentN).
\item{Then, we weight the score of RefSent by the relative relevance in the 
whole training corpus} of this kind of link between document structural 
categories. In order to do so, we consider the document structural category 
previously associated to the citation together with the document structural 
category of the RefSent. For instance, if there is a citance associated to the 
INTRO that references a RefSent belonging to the Abstract and we know that in 
the whole training corpus this situation occurs in 6.5\% of citance-referenced 
sentence pairs, we multiply the RefSent score for 0.065, obtaining the final 
RefSent score.
\item{We choose the 3 reference paper sentences} (RefSents) with the highest 
final RefSent score as the reference paper text spans.
\end{itemize}

\subsection{Algorithm for identifying the discourse facet of the 
											cited text spans}
A linear-kernel SVM classifier was trained to associate each citance with one
 of the five text facets considered in Task 1b. Each citance was characterized 
 by lexical and semantic features extracted from the sentences belonging to 
 the citance context together with the sentences of the reference paper 
 selected as outcome of Task 1a.
Some of the features exploited were:
\begin{itemize}
\item{relative number of sentences belonging to each document structural 
		category}
\item{relative number of sentences belonging to the citance context or 
		reference paper}
\item{relative number of POS}
\item{presence of key lexical patterns}
\end{itemize}

\section{Evaluation and Results}
Two of the teams have submitted their results so far, and the evaluation 
is based on the ROUGE metric \cite{Lin:2004}. ROUGE is a popular evaluation
 method for summarisation systems that compares the text output of the system 
 against a set of target summaries. Since ROUGE uses the actual contents words, 
 and not the offset information, we expect that this metric will give non-zero 
 results for cases when a system chooses a sentence that is similar to, but not 
 exactly, the one chosen by the annotator.

The MQ system and clair\_umich system were both unsupervised, so for the 
evaluation, they were able to use all the data without having to perform 
cross-validation experiments. For the MQ system, the output is the set of 
selected sentences, and the target summaries are the sentences given by the 
annotators.  For the clair\_umich system, the ROUGE-L scores were computed for 
each citing sentence in each annotation file separately and then averaged for 
a topic.


The following paragraphs describe the results for Task 1a, 1b, and the bonus 
Task 2 which was attempted by the MQ system. 

\subsection{Task 1a: For each citance, identify the spans of text 
			(cited text spans) in the RP}

\begin{table*}
\centering
	\begin{tabular}{|l|r|r|r|r|r|r|}
	\hline
	& \multicolumn{3}{|c|}{MQ System} & \multicolumn{3}{|c|}{clair\_umich System}\\
	\hline
	Run & R & P & F1 & R & P & F1\\
	\hline
	Using all features & 0.335 & 0.212 & 0.223 & 0.0 & 0.0 & 0.59\\
	\hline
	\end{tabular}
\caption{ROUGE-L results of the participating systems for task 1a}
\label{tab:task1a}
\end{table*}

Table~\ref{tab:task1av2} shows the ROUGE-L F1 scores of each individual reference document from the SciSumm dataset.

\begin{table*}
  \centering
  \begin{tabular}{|l|r|r|r|r|}
  	\hline
	Paper ID & MQ System & clair\_umich \\
	&	&	System\\
	\hline
	C90-2039 & 0.235 & 0.698\\ 
	C94-2154 & 0.288 & 0.638\\
	E03-1020 & 0.239 & 0.579\\
	H05-1115 & 0.350 & 0.697\\
	H89-2014 & 0.332 & 0.658\\
	J00-3003 & 0.196 & 0.484\\
	J98-2005 & 0.101 & 0.656\\
	N01-1011 & 0.221 & 0.603\\
	P98-1081 & 0.200 & 0.531\\
	X96-1048 & 0.248 & 0.410\\
	\hline
  \end{tabular}
\caption{ROUGE-L F1 results for individual topics 1a}
\label{tab:task1av2}
\end{table*}

\subsection{Task 2: Generate a structured summary of the RP and all of the 
			community discussion of the paper represented in the citances}

The MQ team performed an additional test to see  whether information from 
the citances were useful for building an extractive summary, as is the 
case with the BiomedSumm data \cite{Molla:ALTA2014}. They implemented 
extractive summarisation systems with and without information from the 
citances. 
The summarisers without information from the citances scored each sentence 
as the sum of the \emph{tf.idf} values of the sentence elements. They tried 
the \emph{tf.idf} approach described in Section~ref{sec:tfidf}.

The summarisers with information from the citances scored each candidate 
sentence $i$ on the basis of rank($i$,$c$) obtained in task 1a, which has 
values between 0 (first sentence) and $n$ (last sentence) and represents 
the rank of sentence $i$ in citance $c$:
$$
\hbox{score}(i) = \sum_{c\in\hbox{citances}}1-\frac{\hbox{rank}(i,c)}{n}
$$

The summaries were evaluated using ROUGE-L, where the model summaries are 
the abstract section of the corresponding papers. Since paper X96-1048 of 
the SciSumm data did not have an abstract section, it was removed for this 
experiment.
Table~\ref{tab:task2v2} shows the breakout of ROUGE-L F1 scores per document.

\begin{table*}
  \centering
  \begin{tabular}{|l|r|r|r|r|}
  \hline
	Paper ID & \emph{tf.idf} & task1a  & task1a \\
			&	&	\emph{tf.idf}		& MMR \\
	\hline
	C90-2039\_TRAIN & 0.347 & 0.315 & 0.293\\
	C94-2154\_TRAIN & 0.095 & 0.123 & 0.120\\
	E03-1020\_TRAIN & 0.189 & 0.189 & 0.196\\
	H05-1115\_TRAIN & 0.134 & 0.306 & 0.321\\
	H89-2014\_TRAIN & 0.294 & 0.319 & 0.320\\
	J00-3003\_TRAIN & 0.221 & 0.382 & 0.367\\
	J98-2005\_TRAIN & 0.221 & 0.216 & 0.233\\
	N01-1011\_TRAIN & 0.187 & 0.268 & 0.284\\
	P98-1081\_TRAIN & 0.241 & 0.210 & 0.206\\
	\hline
	Average & 0.214 & 0.259 & 0.260 \\
	\hline
  \end{tabular}
  \caption{ROUGE-L F1 results for summaries generated by the MQ system}
  \label{tab:task2v2}
\end{table*}


\section{Discussion}
\subsection{Comparison with the BioMedSumm task}
Table~\ref{tab:task1a} compares the results of the MQ system's experiments with 
the SciSumm data, against the results from the BiomedSumm data. In all results 
the systems were designed to return 3 sentences, as specified in the shared task. 
All short sentences (under 50 characters) were ignored, to avoid including headings 
or mistakes made by the sentence segmentation algorithm.

\begin{table*}
  \centering
  \begin{tabular}{|l|r|r|r|c|r|r|r|c|}
  	\hline
	& \multicolumn{4}{|c|}{SciSumm} & \multicolumn{4}{|c|}{BiomedSumm}\\
	\hline
	Run & R & P & F1 & CI & R & P & F1 & CI\\
	\hline
    \emph{tf.idf} & 0.316 & 0.198 & 0.211 & 0.185--0.240 & 0.273 &
    0.326 & 0.279 & 0.265--0.293\\
	topics & 0.324 & 0.201 & 0.217 & 0.191--0.245 & 0.288 & 0.357 & 0.300
	& 0.285--0.316\\
	context & 0.339 & 0.214 & 0.225 & 0.197--0.255 & 0.291 & 0.372 & 0.308
	& 0.293--0.323\\
	MMR & 0.335 & 0.212 & 0.223 & 0.195--0.251 &  0.290 & 0.375 & 0.308 & 0.293--0.323\\ 
	\hline
  \end{tabular}
  \caption{ROUGE-L results of the MQ system runs for task 1a}
  \label{tab:task1a}
\end{table*}

The results show an improvement in both domains, with the exception that MMR 
does not improve over the run that uses \emph{tf.idf} over context in SciSumm, 
whereas there is an improvement in BiomedSumm. The absolute values are better 
in the BiomedSumm data, and looking at the confidence intervals it can be 
presumed that the difference between the best and the worst run is 
statistically significant in the BiomedSumm data. The results in the SciSumm 
data are poorer in general and there are no statistically significant 
differences. However, this may be an artifact of the small size of the corpus. 
Overall, the improvement of results in SciSumm mirrors that of the BiomedSumm 
data, so it can be suggested that on adding more information to the models that 
compute \emph{tf.idf}, the results improve. It is expected that alternative 
approaches, which gather related information to be added for computing the 
vector models will produce even better results. The results with MMR appears 
to be contradictory across the two domains but the difference is so small that 
it might not be statistically significant even when we add more evaluation data.


\subsection{Tweaking the Parameters - the clair\_umich Baseline}
For any citing sentence, the TF*IDF cosine similarity was computed with all the 
sentences in the source paper, and any sentences that had a cosine similarity 
higher than a given threshold were added to the matched sentences. 
Table~\ref{tab:clairumichbaseline} shows the precision/recall for different 
values of the cosine threshold:
\begin{table*}
  \centering
  \begin{tabular}{|l|r|r|r|r|}
  	\hline
	Similarity & Precision & Recall & F1-score\\
	Threshold & & & \\
	\hline
	0.01 & 0.027 & 0.641 & 0.051\\
	0.05 & 0.048 & 0.426 & 0.087\\
	0.1 & 0.060 & 0.235 & 0.095\\
	0.2 & 0.079 & 0.081 & 0.080\\
	0.3 & 0.062 & 0.032 & 0.042\\
	0.4 & 0.022 & 0.085 & 0.012\\
	0.5 & 0.007 &  0.002 & 0.003\\
	\hline
  \end{tabular}
  \caption{Precision/Recall for different values of the cosine threshold 
  			for the baseline clair\_umich system}
  \label{tab:clairumichbaseline}
\end{table*}

The F1-score seems to reach a maxima at a similarity threshold of about 0.1. 
The recall at the threshold of 0.1 is about 0.23, while the precision is only 
0.06. This suggests that initial progress can be made on this problem by first 
removing these spurious matches that have high lexical similarity. We present 
some error analysis in the next section.

\paragraph{Error Analysis for the Baseline System}
A number of errors made by the baseline system are due to source sentences that 
match the words but differ slightly in their information content. Here is an 
example.

Citing text: ``use the BNC to build a co-occurrence graph for nouns, based on a 
co-occurrence frequency threshold''

\emph{True positives:}
\begin{itemize}
\item{``Following the method in (Widdows and Dorow, 2002), we build a graph in 
	which each node represents a noun and two nodes have an edge between them if they 
	co-occur in lists more than a given number of times.''}
\end{itemize}
\emph{False positives:}
\begin{itemize}
\item{``Based on the intuition that nouns which co-occur in a list are often 
	semantically related, we extract contexts of the form Noun, Noun,... and/or Noun, 
	e.g. ``genomic DNA from rat, mouse and dog''.''}
\item{``To detect the different areas of meaning in our local graphs, we use a 
	cluster algorithm for graphs (Markov clustering, MCL) developed 
	by van Dongen (2000).''}
\item{``The algorithm is based on a graph model representing words and 
	relationships between them.''}
\end{itemize}

Even though the false positive sentences contain the same lexical items 
(nouns, co-occurrence, graph), they differ slightly in the facts presented. 
Detection of such subtle differences in meaning might be challenging for an 
automated system.

Another set of difficult sentences is when the citing sentence says something 
that is implied by the sentence in the source paper. For example:

Citing text: ``The line of our argument below follows a proof provided in ... 
for the maximum likelihood estimator based on nite tree distributions''

\emph{False negatives:}
\begin{itemize}
\item{``We will show that in both cases the estimated probability is tight.''}
\end{itemize}

Here, the citing text mentions a proof from source paper, but to match the 
sentence in the source paper, the system needs to understand that the act of 
showing something in a scientific paper constitutes a proof.

\section{Limitations and Error Analysis}
There were several limitations in the dataset which were identified in the 
process of annotating and parsing the corpus for use by the participating 
systems; these are discussed below.
\begin{itemize}
\item{The use of ``...'' where text spans are snippets}:The use of ``...'' 
	follows the BioMedSumm standard practice of indicating discontiguous texts. 
	In Citation Text and Reference Text fields, the ``...'' means that there is 
	a gap between two text spans (citation spans or reference spans). They may be 
	on different pages, so the gap might be a text. There might be a formula or a 
	figure there, or some text encoding which is not a part of the annotation.
	However, this notation caused mismatches for sentences which used text from 
	different parts of the same sentence.
\item{Small size of the training corpus:}The corpus comprised only a training set 
	of 10 topics, each with upto 10 citing documents. In this small dataset, 
	participants were asked to conduct a 10-fold cross validation. The small 
	size of the data set meant that there were no statistically significant 
	results, but significance could only be guessed at, from the overall trend of 
	the data.
\item{Errors in parsing the file:} Some of the older PDF files, when parsed to 
	text or xml, had such as misspelt words, spaces within words, sentences in 
	the wrong place and so on. Unfortunately these errors were OCR parsing errors, 
	and not in our control. It was recommended that the participants should 
	configure their string matching to be lenient enough to tackle such problems.
\item{Errors in citation/reference offset numbers:} In the original annotations, 
	citation/reference offset numbers were character-based, and relative to an xml 
	encoding which was not shared in the task, and did not match with the offset 
	numbers on the text-only, cleaned version of the document. Although the text 
	versions of the source documents were shared with the intention to help the 
	participants, this often made their tasks more difficult if their system was 
	geared towards numerical and not system matching. A solution was found for 
	reference offsets by revising them to sentence id numbers based on available 
	XML files from the clair\_umich system's pre-processing stage; however, the 
	citation offsets remain character-based.
\item{Text encoding: Often, the text was not in UTF-8 format as expected}. Some 
	participating teams, like the UPF, solved this by running the universal 	
	charset tool provided by Google Code over all the text and annotations in order 
	to determine the right file encoding to use. It was found that some of the files 
	were also in WINDOWS-1252 and GB18030 formats.
\item{Errors in file construction:} An automatic, open-source software was used 
	to map the citation annotations from a software, Protege, to a text file. 
	However, participants identified several errors in the output -  especially 
	in cases where there was one-to-many mapping between citations and references. 
	Besides this, several annotation texts had no annotation id 
	(Citance Number field). 
\end{itemize}

\section{Conclusions}

The results of the experiments reported in this paper suggest that information from 
related papers may be useful to find the sentences of the reference paper that best 
match the citances. 

Our experiments also suggest that information from the citances may be useful for 
building an extractive summary. This conclusion is compatible with prior research 
that suggest that, in general, information from citing papers may be useful for 
building summaries, as was stated in the original goals of the BiomedSumm and 
CL-Summ shared tasks.

\bibliographystyle{tac2014}

\bibliography{tac2014}

\end{document}
